\chapter{Conclusão}

Ao fim de todo o desenvolvimento desse trabalho de graduação a fim de recapitulação, observa-se uma solução capaz 
de reconhecer textos em fotografias onde as palavras ou instâncias de texto não necessariamente foram a motivação 
da fotografia e, em geral, não são documentos com texto impresso ou manuscrito em papel. Essas imagens são características 
do problema de \textit{Scene Text Recognition} (STR).

A partir do re-uso de um detector e um reconhecedor de texto, populares em trabalhos na área, foi possível implementar 
e avaliar a integração dos dois métodos, obtendo no final uma composição capaz de resolver o problema de STR na 
maneira dita fim-a-fim, agregando as etapas de detecção e reconhecimento de uma vez só, com capacidade de acertar o 
reconhecimento em 70\% das vezes, o que não é o melhor resultado observado no meio acadêmico, mas considerando que 
simplificações foram feitas para possibilitar a implementação, é um resultado satisfatório e cumpre o objetivo do trabalho.

Como observado, a solução desenvolvida apresentou dificuldades nos exemplos de avaliação e no ambiente remoto 
provisionado na nuvem e possibilidades de evolução para trabalhos futuros podem tentar atacá-las, como, por exemplo, 
refinar os parâmetros do modelo de detecção para melhorar os resultados para textos mais longos e textos não-horizontais, 
implementar método de transformação da região de texto para melhorar resultados do reconhecimento, treinar um novo modelo de 
reconhecimento com um dicionário maior de caracteres e até tentar otimizar o uso de recursos computacionais da solução.
