% ---
% RESUMOS
% ---

% RESUMO em português
\setlength{\absparsep}{18pt} % ajusta o espaçamento dos parágrafos do resumo
\begin{resumo}
Nos últimos anos, os termos aprendizado de máquina e aprendizado profundo ficaram bastante populares e já são presentes em soluções disponíveis para o público geral, sendo detecção e reconhecimento de texto um tópico quente nesse contexto, especialmente quando imagens não são documentos.
A partir do levantamento, estudo e revisão da bibliografia sobre \textit{Scene Text Recognition} e os usos de \textit{deep-leraning} nesse problema, este trabalho de graduação propõe o desenvolvimento de uma solução integrada de detecção e reconhecimento de texto de cenas com base em soluções do estado da arte CRAFT e CRNN, publicando uma prova de conceito da solução com uso de computação em nuvem.
A solução avaliada nas bases de dados ICDAR 2011 e ICDAR 2013 atingiu uma média de 70\% de precisão, reconhecendo 7 a cada 10 palavras detectadas. Apesar do relativo sucesso, a solução apresentou limitações no reconhecimento de cenários mais complexos, típicos dos problemas de STR, por exemplo palavras com fontes muito estilizadas, com grande amplitude de tamanho de caracteres e palavras não horizontais, além do alto uso de recursos computacionais.

 \textbf{Palavras-chaves}: Aprendizado de Máquina. Aprendizado Profundo. Redes Neurais Profundas. Reconhecimento Óptico de Caracteres.
\end{resumo}

% ABSTRACT in english
\begin{resumo}[Abstract]
 \begin{otherlanguage*}{english}
   Recently, concepts like machine-learning and deep-learning have become quite popular and have been already been present on customer facing applications. Text detection and recognition problems are also included as many new developments were made in the past years, specially regarding non-document images.
   From the survey and study of the Scene Text Recognition research field and its recent use of deep-learning models, this graduation project presents the development of an end-to-end solution for scene text detection and recognition based on current state of the art methods CRAFT and CRNN, also hosting a proof of concept application for the presented solution using cloud computing. 
   The presented solution was benchmarked using ICDAR 2011 and ICDAR 2013 datasets and has averaged 70\% precision, being able to recognize 7 out of 10 words it had detected. Despite the relative success, the proposed solution has its limitations regarding more difficult STR cases and required high availability of computaional resources.

   \vspace{\onelineskip}
 
   \noindent 
   \textbf{Keywords}: Machine-Learning. Deep-Learning. Deep neural Networks. Optical Character Recognition.
 \end{otherlanguage*}
\end{resumo}