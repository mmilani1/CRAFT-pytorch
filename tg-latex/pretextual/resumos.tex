% ---
% RESUMOS
% ---

% RESUMO em português
\setlength{\absparsep}{18pt} % ajusta o espaçamento dos parágrafos do resumo
\begin{resumo}
Nos últimos anos, os termos, aprendizado de máquina e aprendizado profundo ficaram populares e já são 
presentes em soluções disponíveis para o público geral, sendo detecção e reconhecimento de texto um tópico 
quente nesse contexto, especialmente quando imagens não são documentos.
Este trabalho de graduação propõe o desenvolvimento de uma solução integrada de detecção e reconhecimento 
de texto de cenas com base em soluções do estado da arte, respectivamente, CRAFT (\textit{Character Region Awareness for Text Detection}) 
e CRNN (\textit{Convolutional Recurrent Neural Networks}), com uma prova de conceito da solução com uso de computação em nuvem.
A solução avaliada nas bases de dados ICDAR 2011 e ICDAR 2013 atingiu uma média de 70\% de precisão. 
Apesar do relativo sucesso, a solução apresentou limitações no reconhecimento de cenários mais complexos, típicos 
dos problemas de detecção e reconhecimento de texto em cenas, por exemplo, palavras com fontes muito estilizadas, 
com grande amplitude de tamanho de caracteres e palavras não horizontais, além do alto uso de recursos computacionais.

 \textbf{Palavras-chave}: Aprendizado de Máquina. Aprendizado Profundo. Redes Neurais Profundas. Reconhecimento Óptico de Caracteres.
\end{resumo}

% ABSTRACT in english
\begin{resumo}[Abstract]
 \begin{otherlanguage*}{english}
   Recently, concepts like machine-learning and deep-learning have become quite popular and have been already been 
   present on customer facing applications. Text detection and recognition problems are also included, as many new 
   developments were made in the past years, specially regarding non-document images.
   This graduation project presents the development of an end-to-end solution for scene text detection and recognition 
   based on current state-of-the-art methods CRAFT (Character Region Awareness for Text Detection) and CRNN 
   (Convolutional Recurrent Neural Network), also hosting a proof of concept application for the presented solution using cloud computing. 
   The presented solution was benchmarked using ICDAR 2011 and ICDAR 2013 datasets and has averaged 70\% precision. Despite the 
   relative success, the proposed solution has its limitations regarding more difficult scene text detection and recognition 
   cases and required high availability of computational resources.

   \vspace{\onelineskip}
 
   \noindent 
   \textbf{Keywords}: Machine-Learning. Deep-Learning. Deep Neural Networks. Optical Character Recognition.
 \end{otherlanguage*}
\end{resumo}